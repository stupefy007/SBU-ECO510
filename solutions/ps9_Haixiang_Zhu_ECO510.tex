\documentclass{article}
\usepackage{mathtools}
\usepackage{amssymb}
\usepackage{amsthm}
\usepackage{extarrows}
\usepackage{graphicx}
\usepackage{subcaption}
\usepackage{enumitem}
\title{Assignment 9}
\date{November 9, 2020}
\author{Haixiang Zhu}
\begin{document}
    \maketitle
    \renewcommand{\arraystretch}{1.5}
    \begin{enumerate}
        \item Period-by-period budget constraint for each consumer in period t
        \[c_t+q_tb_{t+1}=b_t+w_t\]
        Given a sequence of endowments $\{\{w_{i,t}\}_{t=0}^\infty\}_{i=1}^2$, a competitive equilibrium with sequential trade consists of sequences of allocations $\{\{c_{i,t}^\ast,b_{i,t+1}^\ast\}_{t=0}^\infty\}_{i=1}^2$ and a sequence of prices $\{(q_{t}^b)^\ast\}_{t=0}^\infty$ such that
        \begin{enumerate}
            \item Given the price system, the allocation solves each consumer's problem. For $i=1,2$
            \begin{align*}
                &\{c_{i,t}^\ast,b_{i,t+1}^\ast\}_{t=0}^\infty=\arg\max_{\{c_{i,t},b_{i,t+1}\}_{t=0}^\infty}\sum_{t=0}^\infty\beta^t\frac{c_{i,t}^{1-\sigma}}{1-\sigma}\\
                &\begin{array}{r@{\quad}r@{}l@{\quad}l}
                s.t.&c_{i,t}+(q_{t}^b)^\ast b_{i,t+1}&= b_{i,t}+w_{i,t}&\forall t\\
                &c_{i,t}&\ge0 &\forall t\\
                &b_{i,0}&=0\\
                &\lim\limits_{T\to\infty}b_{i,T+1}^\ast\prod\limits_{t=0}^T(q_{t}^b)^\ast&\ge0
                \end{array} 
            \end{align*}
            \item All markets clear. For goods market 
            \begin{align*}
                &\sum_i c_{i,t}^\ast=\sum_i w_{i,t}&\forall t\\
                \intertext{For asset market}
                &\sum_i b_{i,t+1}^\ast=0&\forall t
            \end{align*}
        \end{enumerate}
        \item All conditions for equilibrium (FOC+B.C+M.C+TVC+nPg) $\forall t,i$
        \begin{align}
            \beta^t(c_{i,t}^\ast)^{-\sigma}&=\mu_{i,t}\tag{$c_{i,t}$}\\
            (q_{t}^b)^\ast\mu_{i,t}&=\mu_{i,t+1}\tag{$b_{i,t+1}$}\\
            c_{i,t}+(q_{t}^b)^\ast b_{i,t+1}&=b_{i,t}+w_{i,t}\tag{B.C}\\
            c_{1,t}^\ast+c_{2,t}^\ast&=w_{1,t}+w_{2,t}\tag{goods}\\
            b_{1,t+1}^\ast+b_{2,t+1}^\ast&=0\tag{asset}\\
            \lim_{T\to\infty}\beta^T(c_{i,T}^\ast)^{-\sigma}b_{i,T+1}^\ast&\le0\tag{TVC}\\
            \lim_{T\to\infty}b_{i,T+1}^\ast\prod_{t=0}^T(q_{t}^b)^\ast&\ge0\tag{nPg}
        \end{align}
        where $\mu_{i,t}$ is the multiplier on consumer $i$'s budget constraint for each period $t$. The non-negativity constrants is ignored because of the Inada condition for utility function. 
        \item \begin{proof}
            The equivalence between the date-0 equilibrium and the sequential equilibrium.\\
            Recall the characterization of the date-0 trade equilibrium $\forall t,i$
            \[\begin{dcases}
                \beta^t (c_{i,t}^\ast)^{-\sigma}=\lambda_ip_t^\ast\\
                \sum_{t=0}^\infty p_t^\ast c_{i,t}=\sum_{t=0}^\infty p_t^\ast w_{i,t}\\
                \sum_i c_{i,t}^\ast=\sum_i w_{i,t}
            \end{dcases}\]
            where $\lambda_i$ is the multiplier on consumer $i$'s budget constraint.
            \begin{enumerate}
                \item Necessity (only if)\\
                Consider any consumption choice \(\{c_{i,t}^\ast\}_{t=0}^\infty\) that is feasible in the sequential trade equilibrium. Then, by rolling forward the sequential trade budget and using TVC and nPg it can shown that it satisfies the date-0 trade budget
                \begin{align*}
                    b_{i,0}&=c_{i,0}^\ast-w_{i,0}+(q_{0}^b)^\ast b_{i,1}^\ast\\
                    &=c_{i,0}^\ast-w_{i,0}+(q_{0}^b)^\ast(c_{i,1}^\ast-w_{i,1})+(q_{0}^b)^\ast(q_{1}^b)^\ast b_{i,2}^\ast\\
                    &=\dots\\
                    &=\sum_{t=0}^\infty\prod_{s=0}^{t-1}(q_{s}^b)^\ast(c_{i,t}^\ast-w_{i,t})+\lim_{T\to\infty}b_{i,T+1}^\ast\prod_{t=0}^T(q_{t}^b)^\ast\\
                \end{align*}
                Now note that in equilibrium \(\prod\limits_{s=0}^{t-1}(q_{s}^b)^\ast=\beta^t\left(\frac{c_{i,t}^\ast}{c_{i,0}^\ast}\right)^{-\sigma}\) and this also corresponds to \(\dfrac{p_t^\ast}{p_0^\ast}\) in the date-0 trade equilibrium.
                \[b_{i,0}=\sum_{t=0}^\infty\frac{p_t^\ast(c_{i,t}^\ast-w_{i,t})}{p_0^\ast}+\lim_{T\to\infty}b_{i,T+1}^\ast\beta^T\left(\frac{c_{i,T}^\ast}{c_{i,0}^\ast}\right)^{-\sigma}\]
                The last term is $0$ (by using the nPg and TVC conditions) and using the zero initial wealth assumption and the normalization $p_0^\ast=1$ we obtain
                \[0=\sum_{t=0}^\infty p_t^\ast(c_{i,t}^\ast-w_{i,t})\]
                \item Sufficiency (if)\\
                To show the opposite statement, that if \(\{c_{i,t}^\ast\}_{t=0}^\infty\) satisfies the date-0 trade
                budget then it is feasible in the sequential trade economy, one can proceed by constructing the asset trades required to ensure the same consumption. Since in equilibrium
                \[(q_{t}^b)^\ast=\beta\left(\frac{c_{i,t+1}^\ast}{c_{i,t}^\ast}\right)^{-\sigma}\]
                then the asset choices can be constructed recursively
                \[b_{i,t+1}^\ast=\frac{(c_{i,t}^\ast)^{-\sigma}}{\beta(c_{i,t+1}^\ast)^{-\sigma}}(b_{i,t}^\ast+w_{i,t}-c_{i,t}^\ast)\quad\forall t\]
                With these choices for assets, and given that goods' markets clear, the asset market clears in every period (simply add $b_{i,t+1}$ across agents
                and show it equals zero). We can also show in a manner identical to before that
                \[0=b_{i,0}=\sum_{t=0}^\infty p_t^\ast(c_{i,t}^\ast-w_{i,t})+\lim_{T\to\infty}b_{i,T+1}^\ast\beta^T\left(\frac{c_{i,T}^\ast}{c_{i,0}^\ast}\right)^{-\sigma}\]
                and since the date-0 budget constraint is satisfied for this consumption sequence, this implies that
                \[\lim_{T\to\infty}b_{i,T+1}^\ast\beta^T({c_{i,T}^\ast})^{-\sigma}=0\]
                that is, the nPg and TVC conditions are satisfied.
            \end{enumerate}
        \end{proof}
        \item Special cases\\
        Since date-0 equilibrium and sequential equilibrium are equivalent, from FOC
        \[(q_{t}^b)^\ast=\beta\left(\frac{c_{i,t+1}^\ast}{c_{i,t}^\ast}\right)^{-\sigma}=\frac{p_{t+1}^\ast}{p_{t}^\ast}\]
        Then
        \begin{equation}
            c_{i,t}^\ast+\frac{p_{t+1}^\ast}{p_{t}^\ast}b_{i,t+1}^\ast= b_{i,t}^\ast+w_{i,t}\label{eq:SCE}
        \end{equation}
        \begin{enumerate}
            \item $w_{1,t}=2y,w_{2,t}=y\enspace\forall t$\\
            Since $c_{i,t}^\ast=w_{i,t}\enspace\forall t,i$, no trade happens in this case, which implies $b_{i,t}=0\enspace\forall t$
            \item $w_{1,t}=\{2y,y,2y,y,\dots\},w_{2,t}=\{y,2y,y,2y,\dots\}\enspace\forall t$
            \begin{equation}
                \begin{dcases}
                    p_t^\ast=\beta^t\\
                    c_{1,t}^\ast=\frac{2+\beta}{1+\beta}y\\
                    c_{2,t}^\ast=\frac{1+2\beta}{1+\beta}y
                \end{dcases}\quad\forall t\label{sol1}
            \end{equation}
            Plugging \(\eqref{sol1}\) into \(\eqref{eq:SCE}\)
            \begin{align*}
                &b_{1,1}^\ast=\frac{1}{\beta}(0+2y-\frac{2+\beta}{1+\beta}y)=\frac{1}{1+\beta}y=-b_{2,1}^\ast\\
                &b_{1,2}^\ast=\frac{1}{\beta}(b_{1,1}^\ast+y-\frac{1+2\beta}{1+\beta}y)=0=-b_{2,2}^\ast\\
                &\cdots
            \end{align*}
            Therefore,
            \[\begin{dcases}
                b_{1,2t}^\ast=0\\
                b_{1,2t+1}^\ast=\frac{1}{1+\beta}y\\
                b_{2,2t}^\ast=0\\
                b_{2,2t+1}^\ast=-\frac{1}{1+\beta}y
            \end{dcases}\quad\forall t\]
            \item $w_{1,t}=2y,w_{2,t}=\{y,3y,y,3y,\dots\}\enspace\forall t$
            \begin{equation}
                \begin{dcases}
                    p_{2t}^\ast=\beta^{2t}\\
                    p_{2t+1}^\ast=\beta^{2t+1}\left(\frac{5}{3}\right)^{-\sigma}\\
                    c_{1,2t}^\ast=6y\frac{1+\beta\left(\frac{5}{3}\right)^{-\sigma}}{3+5\beta\left(\frac{5}{3}\right)^{-\sigma}}\\
                    c_{1,2t+1}^\ast=10y\frac{1+\beta\left(\frac{5}{3}\right)^{-\sigma}}{3+5\beta\left(\frac{5}{3}\right)^{-\sigma}}\\
                    c_{2,2t}^\ast=3y\frac{1+3\beta\left(\frac{5}{3}\right)^{-\sigma}}{3+5\beta\left(\frac{5}{3}\right)^{-\sigma}}\\
                    c_{2,2t+1}^\ast=5y\frac{1+3\beta\left(\frac{5}{3}\right)^{-\sigma}}{3+5\beta\left(\frac{5}{3}\right)^{-\sigma}}
                \end{dcases}\quad\forall t\label{sol2}
            \end{equation}
            Plugging \(\eqref{sol2}\) into \(\eqref{eq:SCE}\)
            \begin{align*}
                &b_{1,1}^\ast=\frac{1}{\beta}\left(\frac{5}{3}\right)^{\sigma}\left(0+2y-6y\frac{1+\beta\left(\frac{5}{3}\right)^{-\sigma}}{3+5\beta\left(\frac{5}{3}\right)^{-\sigma}}\right)=\frac{4y}{3+5\beta\left(\frac{5}{3}\right)^{-\sigma}}=-b_{2,1}^\ast\\
                &b_{1,2}^\ast=\frac{1}{\beta}\left(\frac{5}{3}\right)^{-\sigma}\left(b_{1,1}^\ast+2y-10y\frac{1+\beta\left(\frac{5}{3}\right)^{-\sigma}}{3+5\beta\left(\frac{5}{3}\right)^{-\sigma}}\right)=0=-b_{2,2}^\ast\\
                &\cdots
            \end{align*}
            Therefore,
            \[\begin{dcases}
                b_{1,2t}^\ast=0\\
                b_{1,2t+1}^\ast=\frac{4y}{3+5\beta\left(\frac{5}{3}\right)^{-\sigma}}\\
                b_{2,2t}^\ast=0\\
                b_{2,2t+1}^\ast=-\frac{4y}{3+5\beta\left(\frac{5}{3}\right)^{-\sigma}}
            \end{dcases}\quad\forall t\]
        \end{enumerate}
        \item $c,b,q,w$ donote the comsumption, bonds agent bought, bond price and exogenous endowment in current period respectively.\\
        \(q',w'\) denote the bond price and endowment in the next period respectively.\\ 
        $b^{-1}$ donotes the amount of bond agent bought in the last period.\\
        Bellman Equation is
        \begin{align*}
            &V(b^{-1},w,q)=\max_{c,b}\left\{\frac{c^{1-\sigma}}{1-\sigma}+\beta V(b,w',q')\right\}\\
            &\begin{array}{r@{\quad}r@{}l}
            s.t.&c+qb\,&=b^{-1}+w\\
            &c\,&\ge0\\
            &q'\,&=f^q(q)\\
            &w'\,&=f^w(w)\\
            &\text{nPg}&\text{ holds}\\
            &b^{-1},w &\text{ given}
            \end{array} 
        \end{align*}
        where state variables are $b^{-1},w,q$ and control variables are $c,b$.
    \end{enumerate}
\end{document}
