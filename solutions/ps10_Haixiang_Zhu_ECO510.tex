\documentclass{article}
\usepackage{mathtools}
\usepackage{amssymb}
\usepackage{amsthm}
\usepackage{extarrows}
\usepackage{graphicx}
\usepackage{subcaption}
\usepackage{enumitem}
\DeclareMathOperator*{\argmax}{arg\,max}
\title{Assignment 10}
\date{November 17, 2020}
\author{Haixiang Zhu}
\begin{document}
    \maketitle
    \renewcommand{\arraystretch}{1.5}
    \begin{enumerate}
        \item 
        \begin{enumerate}
            \item Homogeneous HH
            \begin{enumerate}
                \item Representative HH\\
                Given initial allocations of capital and assets, $k_0$ and \(b_{-1}\), a competitive equilibrium with sequential trade consists of sequences of allocations $\{c_{t}^\ast,b_{t}^\ast,k_{t+1}^\ast,n_{t}^\ast\}_{t=0}^\infty$ and sequences of prices $\{(q_{t}^b)^\ast,r_{t}^\ast,w_{t}^\ast\}_{t=0}^\infty$ such that
                \begin{enumerate}[label=\arabic*)]
                    \item Given prices, allocations are optimal for the household
                    \begin{align*}
                        &\{c_{t}^\ast,b_{t}^\ast,k_{t+1}^\ast,n_{t}^\ast\}_{t=0}^\infty=\argmax_{\{c_{t},b_{t},k_{t+1},n_t\}_{t=0}^\infty}\sum_{t=0}^\infty\beta^tu(c_t,1-n_t)\\
                        &\begin{array}{r@{\quad}r@{}l@{\quad}l}
                        s.t.&c_{t}+(q_{t}^b)^\ast b_t+k_{t+1}-(1-\delta)k_t&= b_{t-1}+w_{t}^\ast n_t+r_t^\ast k_t&\forall t\\
                        &c_{t}&\ge0 &\forall t\\
                        &k_{t+1}&\ge0 &\forall t\\
                        &0\le n_{t}&\le1 &\forall t\\
                        &b_{-1}&=0\\
                        &\lim\limits_{T\to\infty}b_{T}^\ast\prod\limits_{t=0}^T(q_{t}^b)^\ast&\ge0\\
                        &k_0&\text{ given}
                        \end{array} 
                    \end{align*}
                    \item Given prices, allocations are optimal for the firm
                    \[\{k_{t}^\ast,n_{t}^\ast\}=\argmax_{k_t,n_t}F(k_t,n_t)-w_{t}^\ast n_t-r_t^\ast k_t\quad\forall t\]
                    \item All markets clear. For goods market 
                    \begin{align*}
                        c_{t}^\ast+k_{t+1}^\ast-(1-\delta)k_t^\ast&=F(k_t^\ast,n_t^\ast)&\forall t\\
                        \intertext{For asset market}
                        b_{t}^\ast&=0&\forall t
                    \end{align*}
                \end{enumerate}
                \item 
                \begin{proof}
                    If \(k_{t+1}=0\), then return of capital \(r_{t+1}^\ast=F_k(k_{t+1},n_{t+1})\) goes to infinity because of the Inada condition of production function.
                    So it is never optimal for HH to choose \(k_{t+1}=0\), i.e. non-negativity of capital will not bind in equilibrium.
                \end{proof}
                \item FOC
                \begin{align}
                    \beta^tu_c(c_t,1-n_t)&=\lambda_{t}\tag{$c_{t}$}\label{c}\\
                    (q_{t}^b)^\ast\lambda_{t}&=\lambda_{t+1}\tag{$b_{t}$}\label{b}\\
                    \beta^tu_n(c_t,1-n_t)&=-w_t^\ast\lambda_{t}\tag{$n_{t}$}\label{n}\\
                    \lambda_{t}&=(1-\delta+r_{t+1}^\ast)\lambda_{t+1}\tag{$k_{t+1}$}\label{k}
                \end{align}
                where $\lambda_{t}$ is the multiplier on HH's budget constraint for each period $t$.\\
                From FOC of \(\eqref{c},\eqref{b}\) and \(\eqref{k}\)
                \begin{equation}
                    (q_{t}^b)^\ast=\frac{\beta u_c(c_{t+1},1-n_{t+1})}{u_c(c_t,1-n_t)}=\frac{1}{1-\delta+r_{t+1}^\ast}\label{q}
                \end{equation}
                Therefore, the bond price is inverse to the return on capital.\\
                Incorporating date-0 CE
                \begin{equation*}
                    (q_{t}^b)^\ast=\frac{1}{1-\delta+r_{t+1}^\ast}=\frac{\beta u_c(c_{t+1},1-n_{t+1})}{u_c(c_t,1-n_t)}=\frac{p_{t+1}^\ast}{p_t^\ast}
                \end{equation*}
                \item From \(\eqref{q}\), we have proved that in the perfect competitive market, we only need one asset to achieve optimum, all the
                other assets are redundant, and the redundant assets can be priced using the single asset. Thus, the effect of introducing additional assets is indifferent.
            \end{enumerate}
            \item Heterogeneous HH
            \begin{enumerate}
                \item Given initial allocations of capital and assets, $k_{i,0}$ and \(b_{i,-1}\), a competitive equilibrium with sequential trade consists of
                sequences of prices $\{(q_{t}^b)^\ast,r_{t}^\ast,w_{t}^\ast\}_{t=0}^\infty$,
                HH's allocations \(\{\{c_{i,t}^\ast,b_{i,t}^\ast,k_{i,t+1}^\ast,n_{i,t}^\ast\}_{t=0}^\infty\}_{i=1}^2\) and
                firm's choices \(\{K_{t}^\ast,N_{t}^\ast\}_{t=0}^\infty\) such that
                \begin{enumerate}[label=\arabic*)]
                    \item Given prices, allocations are optimal for the household. For $i=1,2$
                    \begin{align*}
                        &\{c_{i,t}^\ast,b_{i,t}^\ast,k_{i,t+1}^\ast,n_{i,t}^\ast\}_{t=0}^\infty=\argmax_{\{c_{i,t},b_{i,t},k_{i,t+1},n_{i,t}\}_{t=0}^\infty}\sum_{t=0}^\infty\beta^tu(c_{i,t},1-n_{i,t})\\
                        &\begin{array}{r@{\quad}r@{}l@{\quad}l}
                        s.t.&c_{i,t}+(q_{t}^b)^\ast b_{i,t}+k_{i,t+1}-(1-\delta)k_{i,t}&= b_{i,t-1}+w_{t}^\ast n_{i,t}+r_t^\ast k_{i,t}&\forall t\\
                        &c_{i,t}&\ge0 &\forall t\\
                        &k_{i,t+1}&\ge0 &\forall t\\
                        &0\le n_{i,t}&\le1 &\forall t\\
                        &b_{i,-1}&=0\\
                        &\lim\limits_{T\to\infty}b_{i,T}^\ast\prod\limits_{t=0}^T(q_{t}^b)^\ast&\ge0\\
                        &k_{i,0}&\text{ given}
                        \end{array} 
                    \end{align*}
                    \item Given prices, allocations are optimal for the firm
                    \[\{K_{t}^\ast,N_{t}^\ast\}=\argmax_{K_{t},N_{t}}F(K_{t},N_{t})-w_{t}^\ast N_{t}-r_t^\ast K_{t}\quad\forall t\]
                    \item All markets clear. For goods market 
                    \begin{align*}
                        \sum_i[c_{i,t}^\ast+k_{i,t+1}^\ast-(1-\delta)k_{i,t}^\ast]&=F(K_{t}^\ast,N_{t}^\ast)&\forall t\\
                        \intertext{For asset market}
                        \sum_ib_{i,t}^\ast&=0&\forall t\\
                        \intertext{For labor market}
                        \sum_in_{i,t}^\ast&=N_{t}^\ast&\forall t\\
                        \intertext{For capital market}
                        \sum_ik_{i,t}^\ast&=K_{t}^\ast&\forall t
                    \end{align*}
                \end{enumerate}
                \item 
                \begin{proof}
                    If \(k_{1,t+1}=k_{2,t+1}=0\), then \(k_{t+1}=0\). Return of capital \(r_{t+1}^\ast=F_k(k_{t+1},n_{t+1})\) goes to infinity because of the Inada condition of production function.
                    So it is never optimal for HH to choose \(k_{1,t+1}=k_{2,t+1}=0\).\\
                Let\((r_{t+1}^b)^\ast\)be the return of bond. In equilibrium, no-arbitrage condition
                \begin{align*}
                    r_{t+1}^\ast+1-\delta&=(r_{t+1}^b)^\ast&\forall t
                \end{align*}
                which implies the capital multiplier \(\nu_{i,t}=0\), i.e. non-negativity of capital will not bind for one of them only.
                \end{proof}
                This would not be true if there were no financial asset available because heterogeneous HH would smooth their consumption via bond market in equilibrium.
                Then the best each HH could do is \(k_{i,t+1}=0\).
                \item Given initial allocations of capital holdings $k_{i,0}$, a competitive equilibrium with date-0 trade consists of 
                sequences of prices \(\{p_{t}^\ast,r_{t}^\ast,w_{t}^\ast\}_{t=0}^\infty\),
                HH's allocations \(\{\{c_{i,t}^\ast,k_{i,t+1}^\ast,n_{i,t}^\ast\}_{t=0}^\infty\}_{i=1}^2\) and
                firm's choices \(\{K_{t}^\ast,N_{t}^\ast\}_{t=0}^\infty\) such that
                \begin{enumerate}[label=\arabic*)]
                    \item Given prices, allocations are optimal for the household. For $i=1,2$
                    \begin{align*}
                        &\{c_{i,t}^\ast,k_{i,t+1}^\ast,n_{i,t}^\ast\}_{t=0}^\infty=\argmax_{\{c_{i,t},k_{i,t+1},n_{i,t}\}_{t=0}^\infty}\sum_{t=0}^\infty\beta^tu(c_{i,t},1-n_{i,t})\\
                        &\begin{array}{r@{\quad}r@{}l@{\quad}l}
                        s.t.&\sum\limits_{t=0}^\infty p_t^\ast[c_{i,t}+k_{i,t+1}-(1-\delta)k_{i,t}]&=\sum\limits_{t=0}^\infty p_t^\ast[w_{t}^\ast n_{i,t}+r_t^\ast k_{i,t}]\\
                        &c_{i,t}&\ge0 &\forall t\\
                        &k_{i,t+1}&\ge0 &\forall t\\
                        &0\le n_{i,t}&\le1 &\forall t\\
                        &k_{i,0}&\text{ given}
                        \end{array} 
                    \end{align*}
                    \item Given prices, allocations are optimal for the firm
                    \[\{K_{t}^\ast,N_{t}^\ast\}=\argmax_{K_{t},N_{t}}F(K_{t},N_{t})-w_{t}^\ast N_{t}-r_t^\ast K_{t}\quad\forall t\]
                    \item All markets clear.
                    \begin{align*}
                        \sum_i[c_{i,t}^\ast+k_{i,t+1}^\ast-(1-\delta)k_{i,t}^\ast]&=F(K_{t}^\ast,N_{t}^\ast)&\forall t\\
                        \sum_in_{i,t}^\ast&=N_{t}^\ast&\forall t\\
                        \sum_ik_{i,t}^\ast&=K_{t}^\ast&\forall t\\
                    \end{align*}
                \end{enumerate}
                \item 
                \begin{proof}
                    Equilibrium conditions \(\forall t,i\)
                    \begin{align}
                        \beta^tu_{c}(c_{i,t},1-n_{i,t})&=\mu_ip_t^\ast\tag{\(c_{i,t}\)}\label{C}\\
                        p_{t+1}^\ast(1-\delta+r_{t+1}^\ast)&=p_t^\ast\tag{\(k_{i,{t+1}}\)}\label{K}\\
                        \beta^tu_{n}(c_{i,t},1-n_{i,t})+\mu_ip_t^\ast w_t^\ast&=0\tag{\(n_{i,t}\)}\label{N}\\
                        \lim\limits_{T\to\infty}\beta^Tu_{c}(c_{i,T},1-n_{i,T})k_{i,T+1}&=0\tag{TVC}\label{tvc}
                    \end{align}
                    where $\mu_i$ is the multiplier on HH i's budget constraint.\\
                    Assume that \(p_0^\ast=1\). From FOC of \(\eqref{C}\)
                    \begin{equation}
                        p_t^\ast=\beta^t\frac{u_{c_{i,t}}}{u_{c_{i,0}}}\label{P}
                    \end{equation}
                    Rewrite budget constraint
                    \begin{align*}
                        \sum_{t=0}^\infty p_t^\ast c_{i,t}=&\sum_{t=0}^\infty p_t^\ast w_t^\ast n_{i,t}+ \sum_{t=0}^\infty p_t^\ast[(1-\delta+r_t^\ast)k_{i,t}-k_{i,t+1}]\\
                        =&\sum_{t=0}^\infty p_t^\ast w_t^\ast n_{i,t}+(1-\delta+r_0^\ast)k_{i,0}-p_0^\ast k_{i,1}+p_0^\ast k_{i,1}-\\
                        &\cdots+\frac{1}{\mu_i}\lim_{T\to\infty}\beta^Tu_{c}(c_{i,T},1-n_{i,T})k_{i,T+1}\tag{using FOC of \(\eqref{K},\eqref{C}\)}
                    \end{align*}
                    Substituting \(p_t^\ast\) with \(\eqref{P}\) and using \(\eqref{tvc}\) condition, we obtain
                    \begin{equation}
                        \sum_{t=0}^\infty\beta^t\frac{u_{c_{i,t}}}{u_{c_{i,0}}}c_{i,t}=\sum_{t=0}^\infty\beta^t\frac{u_{c_{i,t}}}{u_{c_{i,0}}}w_t^\ast n_{i,t}+(1-\delta+r_0^\ast)k_{i,0}
                    \end{equation}
                \end{proof}
                \item 
                \begin{proof}
                    if \(\{c_{t}^\ast,n_{t}^\ast,k_{t+1}^\ast\}_{t=0}^\infty\) satisfies the date-0 trade budget then it is feasible in the sequential trade economy. Since in equilibrium
                    \[(q_{t}^b)^\ast=\beta\frac{u_{c_{i,t+1}^\ast}}{u_{c_{i,t}^\ast}}\]
                    then the asset choices can be constructed recursively
                    \[b_{i,t}^\ast=\frac{u_{c_{i,t}^\ast}}{\beta u_{c_{i,t+1}^\ast}}[b_{i,t-1}^\ast+w_{t}^\ast n_{i,t}^\ast+(1-\delta+r_{t}^\ast)k_{i,t}^\ast-k_{i,t+1}^\ast-c_{i,t}^\ast]\quad\forall t\]
                    Together with \(b_{1,t}+b_{2,t}=0\)
                    \[0=b_{i,-1}=\sum_{t=0}^\infty p_t^\ast[c_{i,t}^\ast+k_{i,t+1}^\ast-(1-\delta)k_{i,t}^\ast-w_{t}^\ast n_{i,t}^\ast-r_{t}^\ast k_{i,t}^\ast]+\lim_{T\to\infty}b_{i,T}^\ast\beta^T\frac{u_{c_{i,T}^\ast}}{u_{c_{i,0}^\ast}}\]
                    and since the date-0 budget constraint is satisfied for this consumption sequence, this implies that
                    \[\lim_{T\to\infty}b_{i,T}^\ast\beta^Tu_{c_{i,T}^\ast}=0\]
                    that is, the nPg and TVC conditions are satisfied.
                \end{proof}
                If there were no financial asset available, the proof will fail in that there exists some time t when expenditure is larger than income in date-0 trade, which cannot be achieved in sequential trade.
            \end{enumerate}
        \end{enumerate}
    \end{enumerate}
\end{document}