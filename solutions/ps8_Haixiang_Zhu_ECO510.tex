\documentclass{article}
\usepackage{mathtools}
\usepackage{amssymb}
\usepackage{amsthm}
\usepackage{extarrows}
\usepackage{graphicx}
\usepackage{subcaption}
\usepackage{enumitem}
\DeclareMathOperator*{\argmax}{arg\,max}
\title{Assignment 8}
\date{November 4, 2020}
\author{Haixiang Zhu}
\begin{document}
    \maketitle
    \renewcommand{\arraystretch}{1.5}
    \begin{enumerate}
        \item Given a collection of endowments $\{\{w_{i,t}\}_{t=0}^\infty\}_{i=1}^2$, a competitive equilibrium with date-0 trading is a collection of allocations $\{\{c_{i,t}^\ast\}_{t=0}^\infty\}_{i=1}^2$ and a collection of prices $\{p_{t}^\ast\}_{t=0}^\infty$ such that
        \begin{enumerate}
            \item Given the price system, the allocation solves each consumer's problem. For $i=1,2$ 
            \begin{align*}
                &\{c_{i,t}^\ast\}_{t=0}^\infty=\argmax_{\{c_{i,t}\}_{t=0}^\infty}\sum_{t=0}^\infty\beta^t\frac{c_{i,t}^{1-\sigma}}{1-\sigma}\\
                &\begin{array}{r@{\quad}r@{}l@{\quad}l}
                s.t.&\sum\limits_{t=0}^\infty p_t^\ast c_{i,t}&\le\sum\limits_{t=0}^\infty p_t^\ast w_{i,t}\\
                &c_{i,t}&\ge0 &\forall t  
                \end{array} 
            \end{align*}
            \item The price are such that all markets clear. 
            \begin{equation*}
                \sum_i c_{i,t}^\ast=\sum_i w_{i,t}\quad\forall t 
            \end{equation*}
        \end{enumerate}
        \item All equilibrium conditions
        \begin{align}
            \beta^t (c_{i,t}^\ast)^{-\sigma}&=\lambda_ip_t^\ast&\forall t,i\label{c1}\\
            \sum_i c_{i,t}^\ast&=\sum_i w_{i,t}&\forall t\notag\\
            \sum_{t=0}^\infty p_t^\ast c_{i,t}&=\sum_{t=0}^\infty p_t^\ast w_{i,t}&\forall i\label{c2}
        \end{align}
        where $\lambda_i$ is the multiplier on consumer $i$'s budget constraint. The non-negativity constrants is ignored because utility function satisfies the Inada condition . 
        \item Solution\\
        Assume that $p_0=1$, from \(\eqref{c1}\), we have
        \begin{equation}
            \left(\frac{c_{1,t}^\ast}{c_{2,t}^\ast}\right)^{-\sigma}=\frac{\lambda_1}{\lambda_2}\quad\forall t \label{c3}
        \end{equation}
        and
        \begin{equation}
            (c_{i,0}^\ast)^{-\sigma}=\lambda_i\quad\forall i\label{lambda}
        \end{equation}
        Therefore, together with the fact that $c_{1,t}^\ast+c_{2,t}^\ast=w_{1,t}+w_{2,t}$, we have
        \begin{equation}
            \left\{\begin{aligned}
                &c_{1,t}^\ast=\frac{\lambda_2^{\frac{1}{\sigma}}}{\lambda_1^{\frac{1}{\sigma}}+\lambda_2^{\frac{1}{\sigma}}}(w_{1,t}+w_{2,t})\\
                &c_{2,t}^\ast=\frac{\lambda_1^{\frac{1}{\sigma}}}{\lambda_1^{\frac{1}{\sigma}}+\lambda_2^{\frac{1}{\sigma}}}(w_{1,t}+w_{2,t})
            \end{aligned}\right.
            \quad\forall t \label{frac}
        \end{equation}
        Rewriting \(\eqref{frac}\)
        \begin{equation}
            c_{i,t}^\ast=\frac{c_{1,0}^\ast}{w_{1,0}+w_{2,0}}(w_{1,t}+w_{2,t}) \quad\forall t,i
        \end{equation}
        Again, from \(\eqref{c1}\), for any two periods $s$ and $t$ 
        \begin{equation}
            \frac{\beta^s(c_{i,s}^\ast)^{-\sigma}}{\beta^t (c_{i,t}^\ast)^{-\sigma}}=\frac{p_s^\ast}{p_t^\ast}=\frac{\beta^s(c_{j,s}^\ast)^{-\sigma}}{\beta^t (c_{j,t}^\ast)^{-\sigma}}\quad\forall i,j,s,t\label{p}
        \end{equation}
        Then
        \begin{equation*}
            \frac{c_{1,s}^\ast}{c_{1,t}^\ast}=\frac{c_{2,s}^\ast}{c_{2,t}^\ast}=\frac{c_{1,s}^\ast+c_{2,s}^\ast}{c_{1,t}^\ast+c_{2,t}^\ast}=\frac{w_{1,s}+w_{2,s}}{w_{1,t}+w_{2,t}}
        \end{equation*}
        Therefore, assume that $p_0^\ast=1$
        \begin{align}
            &\frac{p_s^\ast}{p_t^\ast}=\beta^{s-t}\left(\frac{w_{1,s}+w_{2,s}}{w_{1,t}+w_{2,t}}\right)^{-\sigma}\quad\forall t\label{p1}\\
            \Rightarrow&p_t^\ast=\beta^t\left(\frac{w_{1,t}+w_{2,t}}{w_{1,0}+w_{2,0}}\right)^{-\sigma}\quad\forall t \label{p2}
        \end{align}
        From \(\eqref{p}\)
        \begin{equation}
            c_{i,t}^\ast=\left(\frac{p_t^\ast}{\beta^t}\right)^{-\frac{1}{\sigma}}c_{i,0}^\ast\label{ct}
        \end{equation}
        Let $w_{1,0}+w_{2,0}=W_0$ and $w_{1,t}+w_{2,t}=W_t$. Plugging \(\eqref{p2}\) and \(\eqref{ct}\) into \(\eqref{c2}\)
        \begin{align}
            &\sum_{t=0}^\infty p_t^\ast w_{i,t}=\sum_{t=0}^\infty p_t^\ast\left(\frac{p_t^\ast}{\beta^t}\right)^{-\frac{1}{\sigma}}c_{i,0}^\ast\notag\\
            \Rightarrow& c_{i,0}^\ast=\frac{\sum_{t=0}^\infty\left[\beta^t\left(\frac{W_t}{W_0}\right)^{-\sigma}w_{i,t}\right]}{\sum_{t=0}^\infty\left[\beta^t\left(\frac{W_t}{W_0}\right)^{1-\sigma}\right]} \label{c0}
        \end{align}
        Plugging \(\eqref{c0}\) into \(\eqref{frac}\) and \(\eqref{lambda}\)
        \begin{align}
            c_{i,t}^\ast&=\frac{W_t}{W_0}c_{i,0}^\ast=\frac{\sum_{t=0}^\infty\left(\beta^t W_t^{-\sigma}w_{i,t}\right)}{\sum_{t=0}^\infty\left(\beta^tW_t^{1-\sigma}\right)}W_t\notag\\
            \lambda_i&=\left[\frac{\sum_{t=0}^\infty\left(\beta^t W_t^{-\sigma}w_{i,t}\right)}{\sum_{t=0}^\infty\left(\beta^tW_t^{1-\sigma}\right)}W_0\right]^{-\sigma}\label{lambdai}
        \end{align}
        \item Intuitive explanation
        \begin{enumerate}
            \item From \(\eqref{p1}\), WLOG, assume that $s>t$.
            \begin{enumerate}
                \item The smaller the ratio of aggregate endowments $\frac{w_{1,s}+w_{2,s}}{w_{1,t}+w_{2,t}}$ in period $s$, the more scarce the good, the higher the relative value of the good in period $s$ and the higher its relative price.
                \item The larger $\beta$ is, the less discounted the good in period s, the greater the relative value of the good in period $s$ and the greater its relative price. 
                \item The larger $\sigma$ is, the less willing is the consumer to substitute consumption across time, the lower the relative price of the good in period $s$ with more aggregate endowment.
            \end{enumerate}
            \item From \(\eqref{frac}\), this constant fraction is an illustration of consumption smoothing. Variation in individual endowments does not translate to variation in consumption because agents smooth their consumption through trade.
            Other things equal, the larger endowments or earlier endowments or smoother endowments, the larger consumption fraction.
        \end{enumerate}
        \item Special cases
        \begin{enumerate}
            \item In this case, the relative price and consumption are all constant across periods. Perfect consumption smoothing is feasible. 
            Denote constant consumption $\bar{c}_i^\ast$ for each $i$. From $\eqref{p2}$, we have
            \begin{equation*}
                p_t^\ast=\beta^t\quad\forall t
            \end{equation*}
            From budget constraint, we have
            \begin{equation}
                \sum_{t=0}^\infty\beta^t(\bar{c}_i^\ast-w_{i,t})=0\label{bc}
            \end{equation}
            where $w_{i,t}=w_i$. Therefore, the solution is
            \begin{equation*}
                \left\{\begin{aligned}
                    &c_{1,t}^\ast=2y\\
                    &c_{2,t}^\ast=y
                \end{aligned}\right.
                \quad\forall t
            \end{equation*}
            \item Again, aggregate endowment is constant, therefore, the relative price and consumption for each consumer are constant. From \(\eqref{p2}\), we have
            \begin{equation*}
                p_t^\ast=\beta^t\quad\forall t
            \end{equation*}
            From $\eqref{bc}$, we have
            \begin{equation*}
                \left\{\begin{aligned}
                    &\sum_{t=0}^\infty\beta^t\bar{c}_1^\ast=2y\sum_{t=0}^\infty\beta^{2t}+y\sum_{t=0}^\infty\beta^{2t+1}\\
                    &\sum_{t=0}^\infty\beta^t\bar{c}_2^\ast=y\sum_{t=0}^\infty\beta^{2t}+2y\sum_{t=0}^\infty\beta^{2t+1}
                \end{aligned}\right.
            \end{equation*}
            Hence, the soulution is
            \begin{equation*}
                \left\{\begin{aligned}
                    &c_{1,t}^\ast=\frac{2+\beta}{1+\beta}y\\
                    &c_{2,t}^\ast=\frac{1+2\beta}{1+\beta}y
                \end{aligned}\right.
                \quad\forall t
            \end{equation*}
            \item Since aggregate endowment is constant in even and odd periods, therefore, the relative price and consumption for each consumer are constant in even and odd periods. From \(\eqref{p2}\), we have
            \begin{equation*}
                \left\{\begin{aligned}
                    &p_{2t}^\ast=\beta^{2t}\\
                    &p_{2t+1}^\ast=\beta^{2t+1}\left(\frac{5}{3}\right)^{-\sigma}
                \end{aligned}\right.
                \quad\forall t
            \end{equation*}
            From \(\eqref{frac}\), consumption in any period is a constant fraction of aggregate endowment in that period. Therefore, we have
            \begin{equation*}
                \left\{\begin{aligned}
                    &c_{i,2t}^\ast=\mu_i(w_{1,2t}+w_{2,2t})\\
                    &c_{i,2t+1}^\ast=\mu_i(w_{1,2t+1}+w_{2,2t+1})
                \end{aligned}\right.
                \quad\forall t
            \end{equation*}
            where $\mu_i$ is a constant fraction that depends on the endowments and the utility function parameters $\beta$ and $\sigma$. Then
            \begin{equation}
                \frac{\bar{c}_{i,2t}^\ast}{\bar{c}_{i,2t+1}^\ast}=\frac{3}{5}\quad\forall i,t \label{frac1}
            \end{equation}
            From $\eqref{bc}$, we have
            \begin{equation}
                \left\{\begin{aligned}
                    &\sum_{t=0}^\infty\beta^{2t}\bar{c}_{1,2t}^\ast+\sum_{t=0}^\infty\beta^{2t+1}\left(\frac{5}{3}\right)^{-\sigma}\bar{c}_{1,2t+1}^\ast=2y\sum_{t=0}^\infty\beta^{2t}+2y\sum_{t=0}^\infty\beta^{2t+1}\left(\frac{5}{3}\right)^{-\sigma}\\
                    &\sum_{t=0}^\infty\beta^{2t}\bar{c}_{2,2t}^\ast+\sum_{t=0}^\infty\beta^{2t+1}\left(\frac{5}{3}\right)^{-\sigma}\bar{c}_{2,2t+1}^\ast=y\sum_{t=0}^\infty\beta^{2t}+3y\sum_{t=0}^\infty\beta^{2t+1}\left(\frac{5}{3}\right)^{-\sigma}
                \end{aligned}\right.\label{sys1}
            \end{equation}
            Plugging \(\eqref{frac1}\) into \(\eqref{sys1}\), the solution is
            \begin{equation*}
                \left\{\begin{aligned}
                    &c_{1,2t}^\ast=6y\frac{1+\beta\left(\frac{5}{3}\right)^{-\sigma}}{3+5\beta\left(\frac{5}{3}\right)^{-\sigma}}\\
                    &c_{1,2t+1}^\ast=10y\frac{1+\beta\left(\frac{5}{3}\right)^{-\sigma}}{3+5\beta\left(\frac{5}{3}\right)^{-\sigma}}\\
                    &c_{2,2t}^\ast=3y\frac{1+3\beta\left(\frac{5}{3}\right)^{-\sigma}}{3+5\beta\left(\frac{5}{3}\right)^{-\sigma}}\\
                    &c_{2,2t+1}^\ast=5y\frac{1+3\beta\left(\frac{5}{3}\right)^{-\sigma}}{3+5\beta\left(\frac{5}{3}\right)^{-\sigma}}
                \end{aligned}\right.
                \quad\forall t
            \end{equation*}
        \end{enumerate}
        \item FOC for this planner's problem 
        \begin{equation*}
            \xi_i\beta^t(c_{i,t}^\ast)^{-\sigma}=\theta_t\quad\forall t,i
        \end{equation*}
        where $\theta_t$ is the multiplier on the resource constraint at t. Then
        \begin{equation}
            \left(\frac{c_{1,t}^\ast}{c_{2,t}^\ast}\right)^{-\sigma}=\frac{\xi_2}{\xi_1}\label{frac2}
        \end{equation}
        From \(\eqref{c3}\) in competitive equilibrium and \(\eqref{frac2}\), the solution is
        \begin{equation}
            \left\{\begin{aligned}
                &\xi_1=\frac{\lambda_2}{\lambda_1+\lambda_2}\\
                &\xi_2=\frac{\lambda_1}{\lambda_1+\lambda_2}
            \end{aligned}\right.\label{frac3}
        \end{equation}
        where $\lambda_i$ is the multiplier on consumer $i$'s budget constraint.\\
        Plugging \(\eqref{lambdai}\) into \(\eqref{frac3}\)
        \begin{equation*}
            \left\{\begin{aligned}
                &\xi_1=\frac{(\sum_{t=0}^\infty\beta^t W_t^{-\sigma}w_{1,t})^{\sigma}}{(\sum_{t=0}^\infty\beta^t W_t^{-\sigma}w_{1,t})^{\sigma}+(\sum_{t=0}^\infty\beta^t W_t^{-\sigma}w_{2,t})^{\sigma}}\\
                &\xi_2=\frac{(\sum_{t=0}^\infty\beta^t W_t^{-\sigma}w_{2,t})^{\sigma}}{(\sum_{t=0}^\infty\beta^t W_t^{-\sigma}w_{i,t})^{\sigma}+(\sum_{t=0}^\infty\beta^t W_t^{-\sigma}w_{i,t})^{\sigma}}
            \end{aligned}\right.
        \end{equation*}
        According to the specific cases from part 5, intuitively, these weights depend on the endowments and the utility function parameters $\beta$ and $\sigma$.
        Other things equal, higher or earlier endowment and smoother endowment will imply a higher weight.
    \end{enumerate}
\end{document}