\documentclass{article}
\usepackage{mathtools}
\usepackage{amssymb}
\usepackage{amsthm}
\usepackage{extarrows}
\usepackage{graphicx}
\usepackage{subcaption}
\usepackage{enumitem}
\title{Assignment 7}
\date{October 18, 2020}
\author{Haixiang Zhu}
\begin{document}
    \maketitle
    \renewcommand{\arraystretch}{1.2}
    \begin{enumerate}
        \item Population growth
        \begin{enumerate}
            \item Suppose that the economy reached its BGP at step $t=T$, where the growth rate of capital is constant. The growth rates of aggregated consumption, output, investment,
            capital, labor and the fraction of time devoted to work at BGP are denoted by $g_C, g_Y, g_I, g_K, g_E\text{ and }g_n$ respectively. \\
            From the capital accumulation equation, 
            \begin{equation*}
                \frac{K_{t+1}}{K_t}=1-\delta+\frac{I_t}{K_t},\quad\forall t\ge T
            \end{equation*}
            Since the LHS of the equation above is constant, equal to $g_K$, and $(1-\delta)$ is also constant, $\frac{I_t}{K_t}$ must be constant. 
            \begin{equation*}
                \frac{I_t}{K_t}=\frac{I_T}{K_T}\left(\frac{g_I}{g_K}\right)^{t-T},\quad\forall t\ge T
            \end{equation*}
            Since $\frac{I_t}{K_t}$ and $\frac{I_T}{K_T}$ are constant, $\frac{g_I}{g_K}$ must be constant, which implies $g_I=g_K$. \\
            From the resource constraint, 
            \begin{align*}
                &C_t+I_t=Y_t,\quad\forall t\ge T\\
                \Rightarrow&\frac{C_T}{Y_T}\left(\frac{g_C}{g_Y}\right)^{t-T}+\frac{I_T}{Y_T}\left(\frac{g_I}{g_Y}\right)^{t-T}=1,\quad\forall t\ge T
            \end{align*}
            If $g_C<g_Y$ and $g_I<g_Y$, then LHS will be falling to $0$ as t increases.\\
            If $g_C<g_Y$ and $g_I>g_Y$, then LHS will be rising to $\infty$ as t increases.\\
            If $g_C>g_Y$ and $g_I<g_Y$, then LHS will be rising to $\infty$ as t increases.\\
            If $g_C>g_Y$ and $g_I>g_Y$, then LHS will be rising to $\infty$ as t increases.\\
            Therefore, for the resource constraint to hold in a growing economy, none of those cases above are possible and it can only be that $g_C=g_Y=g_I$.\\
            From the production function and the fact that it has constant returns to scale,
            \begin{equation*}
                \frac{Y_t}{K_t}=F(1,\frac{E_tn_t}{K_t})
            \end{equation*}
            Since $Y_t$ and $K_t$ grow at equal rates at BGP, the LHS is constant. Thus $\frac{E_tn_t}{K_t}$ must be constant, which implies $g_E=g_K$ and $g_n=1$ because the fraction of time devoted to work cannot be growing infinitely.\\
            We ignore the trivial condition that $g_n$ could be smaller than $1$, because the marginal production of labor input will be infinite if $n_t=0$. So, it is not rational to have a $g_n$ which is smaller than $1$ at BGP. 
            Because $\dfrac{E_t}{M_t}$ is constant, $g_E=g$. Eventually,
            \begin{equation*}
                \begin{cases}
                    &g_C=g_Y=g_I=g_K=g_E=g\\
                    &g_n=1
                \end{cases}
            \end{equation*}
            \item The planer's problem is 
            \begin{align*}
                &\max_{\{C_t,M_t,K_{t+1},n_t\}_{t=0}^\infty}\sum_{t=0}^\infty\beta^tu(\frac{C_t}{M_t},1-n_t)=\sum_{t=0}^\infty\beta^t\frac{\left(\frac{C_t}{M_t}v(1-n_t)\right)^{1-\sigma}-1}{1-\sigma}\\
                & \begin{array}{@{\qquad\quad}r@{\qquad\quad}r@{}l}
                s.t.&C_t+K_{t+1}-(1-\delta)K_t\,&=F(K_t,E_tn_t)\\
                 &K_0&\enspace\text{given}
                \end{array}
            \end{align*}
            The first order conditions can be summarized in the following Euler equation (intertemporal condition)
            \begin{equation*}
                u_C(\frac{C_t}{M_t},1-n_t)=\beta u_C(\frac{C_{t+1}}{M_{t+1}},1-n_{t+1})[1-\delta+F_K(K_{t+1},E_{t+1}n_{t+1})]
            \end{equation*}
            and an intratemporal condition 
            \begin{equation*}
                -u_n(\frac{C_t}{M_t},1-n_t)=u_C(\frac{C_t}{M_t},1-n_t)E_tF_2(K_t,E_tn_t)
            \end{equation*}
            Plugging $u_C$ and $u_n$,
            \begin{align}
                &\frac{M_{t+1}\left(\frac{C_t}{M_t}v(1-n_t)\right)^{-\sigma}v(1-n_t)}{\beta M_t\left(\frac{C_{t+1}}{M_{t+1}}v(1-n_{t+1})\right)^{-\sigma}v(1-n_{t+1})}-1+\delta=F_K(K_{t+1},E_{t+1}n_{t+1})\label{a}\\
                &\frac{v'(1-n_t)C_t}{v(1-n_t)E_t}=F_2(K_t,E_tn_t)\label{b}
            \end{align}
            Since F is homogeneous of degree 1, $F_K$ and $F_2$ are homogeneous of degree 0, which implies $F_K$ and $F_2$ are constant at BGP and $g_K=g_E$, $g_n=1$.
            Then, in equation \(\eqref{a}\), $\dfrac{C_t}{M_t}$ must be constant, which implies $g_C=g$. Then LHS is equal to $\dfrac{g}{\beta}-1+\delta$. As for equation \(\eqref{b}\), $\dfrac{C_t}{E_t}$ must be constant, which implies $g_C=g_E$.
            So, we have $g_C=g_K=g_E=g$ and $g_n=1$. Obviously, $g_K=g_I=g_Y$.\\
            Hence, $g_C=g_Y=g_I=g_K=g_E=g$ and $g_n=1$ are consistent with the planner's optimality conditions. 
            \item The transformation is
            \begin{align*}
                \left\{\begin{aligned}
                    &c_t=\frac{C_t}{M_t}\\
                    &i_t=\frac{I_t}{M_t}\\
                    &k_t=\frac{K_t}{M_t}\\
                    &y_t=\frac{Y_t}{M_t}
                \end{aligned}\right.
            \end{align*}
            where $c_t$, $i_t$, $k_t$ and $y_t$ are the consumption per capita, the investment per capita, the capital stock per capita and the output per capita respectively.\\
            Since $g_C=g_Y=g_I=g_K=g_E=g$ and $g_n=1$ at BGP, $c_t$, $i_t$, $k_t$ and $y_t$ remain constant in the long run.
            Therefore, the transformed problem is
            \begin{align*}
                &\max_{\{c_t,k_{t+1},n_t\}_{t=0}^\infty}\sum_{t=0}^\infty\beta^t\frac{\left[c_tv(1-n_t)\right]^{1-\sigma}-1}{1-\sigma}\\
                & \begin{array}{@{\qquad}r@{\qquad}r@{}l}
                s.t.&c_t+gk_{t+1}-(1-\delta)k_t\,&=F(k_t,\frac{E_tn_t}{M_t})\\
                 &k_0&\enspace\text{given}
                \end{array}
            \end{align*}
            \item The economy converges to BGP where $g_C=g_Y=g_I=g_K=g_E=g$ and $g_n=1$.
            \begin{itemize}
                \item Since $g_Y=g$, \(\dfrac{Y_t}{M_t}\) is constant. So, GDP per capita does not grow at constant rate.
                \item Since $g_K=g_E$, $\dfrac{K_t}{E_t}$ is constant. So, capital to labor ratio does not grow at constant rate.
                \item Since $g_K=g_Y$, $\dfrac{K_t}{Y_t}$ is constant. So, capital to output ratio is constant.
                \item As demonstrated before, \(F_K\) and \(F_2\) are constant at BGP.
                \[\begin{dcases}
                    r_t=F_K(K_t,E_tn_t)\\
                    w_t=F_2(K_t,E_tn_t)\\
                \end{dcases}\]
                which implies real rates of return are constant.
                \item Since \(g_K=g_E=g_Y\) and real rates of return and real wage rate are constant, \(\dfrac{r_tK_t}{Y_t}\) and \(\dfrac{w_tE_tn_t}{Y_t}\) are constant.
                So, capital and labor shares of total income are constant.
                \item Growth rates vary persistently across countries because the utility function and constraints may different in other countires.
            \end{itemize}
        \end{enumerate}
        \item Bellman Guess-and-Verify
        \begin{enumerate}
            \item The infinite Bellman equation we are trying to solve is
            \begin{align*}
                &V(k)=\max_{c,k'}\{\ln(c)+\beta(A+Bk')\}\\
                &\begin{array}{r@{\quad}l}
                    s.t.&c=k^\alpha-k'\ge0\\
                    &k'\ge0\\
                    &k\enspace\text{given} 
                \end{array}
            \end{align*}
            Lagrangian Function:
            \begin{equation*}
                L=\ln(c)+\beta(A+Bk')+\lambda(k^\alpha-k'-c)+\nu c+\mu k'
            \end{equation*}
            Necessary Conditions:
            \begin{equation*}
                \left\{\begin{aligned}
                    &\frac{1}{c}-\lambda+\nu=0\\
                    &\beta B-\lambda+\mu=0\\
                    &c=k^\alpha-k'\\
                    &c\ge0\\
                    &k'\ge0\\
                    &\lambda\ge0\\
                    &\nu\ge0\\
                    &\mu\ge0\\
                    &\nu c=0\\
                    &\mu k'=0
                \end{aligned}\right.
            \end{equation*}
            Since $u(\cdot)$ satisfy the Inada conditions and if $c\to0$, $\frac{1}{c}\to\infty$, which contradicts the FOC of $[c]$, the consumption constraint will not bind, i.e. $c>0$.
            By the complementary slackness, $\nu=0$.\\
            Reduced form
            \begin{equation*}
                \beta B-\frac{1}{k^\alpha-k'}+\mu=0
            \end{equation*}
            When $k'=\mu=0$,
            \begin{equation*}
                B=\frac{1}{\beta k^\alpha}
            \end{equation*}
            When $k'=0$, $\mu>0$,
            \begin{align*}
                &\mu=\frac{1}{k^\alpha}-\beta B>0\\
                \Rightarrow&B<\frac{1}{\beta k^\alpha}
            \end{align*}
            When $k'>0$, $\mu=0$,
            \begin{align*}
                &k'=k^\alpha-\frac{1}{\beta B}>0\\
                \Rightarrow&B>\frac{1}{\beta k^\alpha}
            \end{align*}
            Hence,
            \begin{align*}
                \left\{\begin{aligned}
                    &k'>0(not\enspace binding) &&B>\frac{1}{\beta k^\alpha}\\
                    &k'=0(binding) &&B\le\frac{1}{\beta k^\alpha}
                \end{aligned}\right.
            \end{align*}
            Let $g(k)$ and $g^c(k)$ be the optimal policy function for capital and consumption respectively.

            If $B>\frac{1}{\beta k^\alpha}$,
            \begin{align*}
                \left\{\begin{aligned}
                    &k'=k^\alpha-\frac{1}{\beta B}\equiv g(k)\\
                    &c=\frac{1}{\beta B}\equiv g^c(k)
                \end{aligned}\right.
            \end{align*}
            If $B\le\frac{1}{\beta k^\alpha}$,
            \begin{align*}
                \left\{\begin{aligned}
                    &k'=0\equiv g(k)\\
                    &c=k^\alpha\equiv g^c(k)
                \end{aligned}\right.
            \end{align*}
            \item If $B>\frac{1}{\beta k^\alpha}$,
            \begin{equation*}
                V^{implied}(k)=-\ln(\beta B)+\beta A+\beta Bk^\alpha-1
            \end{equation*}
            By equating coefficients with $V^{guess}(k)$,
            \begin{align*}
                &\left\{\begin{aligned}
                    &-\ln(\beta B)+\beta A-1=A\\
                    &\beta B=B\\
                    &\alpha=1
                \end{aligned}\right.\\
                \Rightarrow&
                \begin{dcases}
                    \beta=1\\
                    B=e^{-1}
                \end{dcases}
            \end{align*}
            Since $\alpha\in(0,1)$, drop this solution.

            If $B\le\frac{1}{\beta k^\alpha}$,
            \begin{equation*}
                V^{implied}(k)=\alpha\ln(k)+\beta A
            \end{equation*}
            By equating coefficients with $V^{guess}(k)$,
            \begin{align*}
                \left\{\begin{aligned}
                    &\beta A=A\\
                    &B=0\\
                    &\alpha=0
                \end{aligned}\right.
            \end{align*}
            Since $\alpha\in(0,1)$, A and B cannot be found.\\
            The reason why A and B cannot be found is that the guess of functional form is wrong.
        \end{enumerate}
    \end{enumerate}
\end{document}