\documentclass{article}
\usepackage{mathtools}
\usepackage{amssymb}
\usepackage{amsthm}
\usepackage{extarrows}
\usepackage{graphicx}
\usepackage{subcaption}
\usepackage{enumitem}
\DeclareMathOperator*{\argmax}{arg\,max}
\title{Assignment 11}
\date{November 22, 2020}
\author{Haixiang Zhu}
\begin{document}
    \maketitle
    \renewcommand{\arraystretch}{1.5}
    Let \(S_0=\{s_0\},\pi(s_0)=1\)
    \begin{enumerate}
        \item Date-0 trade
        \begin{enumerate}
            \item Given endowment processes and their corresponding probability distributions \(\{\{w_{i,t}(s_t)\}_{s_t\in S_t}\}_{t=0}^1\) for each \(i\), 
            a competitive equilibrium with date-0 trade is a set of allocations \(\{\{c_{i,t}^\ast(s_t)\}_{s_t\in S_t}\}_{t=0}^1\) 
            for each \(i\) and prices \(\{\{p_{t}^\ast(s_t)\}_{s_t\in S_t}\}_{t=0}^1\) such that
            \begin{enumerate}[label=\arabic*)]
                \item Given prices, allocations are optimal for each consumer \(i\)
                \begin{align*}
                    &\{c_{i,t}^\ast(s_t)\}_{t=0}^1=\argmax_{\{c_{i,t}(s_t)\}_{t=0}^1}\sum_{t=0}^1\sum_{s_t\in S_t}\beta^t\pi(s_t)\frac{[c_{i,t}(s_t)]^{1-\sigma}}{1-\sigma}\\
                    &\begin{array}{r@{\quad}r@{}l@{\quad}l}
                    s.t.&\sum\limits_{t=0}^1\sum\limits_{s_t\in S_t}p_t^\ast(s_t)c_{i,t}(s_t)&\,\le\sum\limits_{t=0}^1\sum\limits_{s_t\in S_t}p_t^\ast(s_t)w_{i,t}(s_t)\\
                    &c_{i,t}(s_t)&\,\ge0 &\forall t
                    \end{array} 
                \end{align*}
                \item The price are such that all markets clear.
                \begin{align*}
                    \sum_i c_{i,t}^\ast(s_t)=\sum_i w_{i,t}(s_t)&&\forall t,i\\
                \end{align*}
            \end{enumerate}
            \item All equilibrium conditions
            \begin{align}
                \beta^t\pi(s_t)[c_{i,t}^\ast(s_t)]^{-\sigma}&=\lambda_ip_t^\ast(s_t)&\forall t,i\tag{\(c_{i,t}(s_t)\)}\label{c}\\
                \sum_{t=0}^1\sum\limits_{s_t\in S_t}p_t^\ast(s_t)c_{i,t}(s_t)&=\sum_{t=0}^1\sum\limits_{s_t\in S_t}p_t^\ast(s_t)w_{i,t}(s_t)&\forall i\tag{B.C.}\label{BC}\\
                \sum_i c_{i,t}^\ast(s_t)&=\sum_i w_{i,t}(s_t)&\forall t,i\tag{M.C.}\label{MC}
            \end{align}
            where \(\lambda_i\) is the multiplier on consumer \(i\)'s budget constraint.\\
            From FOC of \(\eqref{c}\), and note that, \(\pi(s_0)=1,p_0^\ast=1,c_{i,0}^\ast(s_0)=c_{i,0}^\ast\)
            \begin{align}
                p_t^\ast(s_t)&=\beta^t\pi(s_t)\left[\frac{c_{i,t}^\ast(s_t)}{c_{i,0}^\ast}\right]^{-\sigma}\label{1}\\
                c_{i,t}^\ast(s_t)&=\left[\frac{p_t^\ast(s_t)}{\beta^t\pi(s_t)}\right]^{-\frac{1}{\sigma}}c_{i,0}^\ast\label{2}
            \end{align}
            From \(\eqref{MC}\) and FOC of \(\eqref{c}\)
            \begin{equation}
                \frac{c_{A,t}^\ast(s_t)}{c_{A,0}}=\frac{c_{B,t}^\ast(s_t)}{c_{B,0}}=\frac{c_{A,t}^\ast(s_t)+c_{B,t}^\ast(s_t)}{c_{A,0}+c_{B,0}}=\frac{w_{A,t}(s_t)+w_{B,t}(s_t)}{w_{A,0}+w_{B,0}}\label{3}
            \end{equation}
            Plugging \(\eqref{3}\) into \(\eqref{1}\), and let \(w_{A,t}(s_t)+w_{B,t}(s_t)=W_t(s_t),w_{A,0}+w_{B,0}=W_0\)
            \begin{align}
                p_t^\ast(s_t)=\beta^t\pi(s_t)\left[\frac{W_t(s_t)}{W_0}\right]^{-\sigma}&,s_t\in S_t\label{4}
            \end{align}
            Plugging \(\eqref{2},\eqref{4}\) into \(\eqref{MC}\)
            \begin{align}
                \sum_{t}\sum_{s_t}p_t^\ast(s_t)\left[\frac{p_t^\ast(s_t)}{\beta^t\pi(s_t)}\right]^{-\frac{1}{\sigma}}c_{i,0}^\ast&=\sum_{t}\sum_{s_t}p_t^\ast(s_t)w_{i,t}(s_t)\notag\\
                c_{i,0}^\ast&=\frac{\sum\limits_{t}\sum\limits_{s_t}\beta^t\pi(s_t)\left[\frac{W_t(s_t)}{W_0}\right]^{-\sigma}w_{i,t}(s_t)}{\sum\limits_{t}\sum\limits_{s_t}\beta^t\pi(s_t)\left[\frac{W_t(s_t)}{W_0}\right]^{1-\sigma}}\notag\\
                c_{i,t}^\ast(s_t)&=\frac{W_t(s_t)}{W_0}c_{i,0}^\ast\notag\\
                c_{i,t}^\ast(s_t)&=\frac{\sum\limits_{t}\sum\limits_{s_t}\beta^t\pi(s_t)\left[W_t(s_t)\right]^{-\sigma}w_{i,t}(s_t)}{\sum\limits_{t}\sum\limits_{s_t}\beta^t\pi(s_t)\left[W_t(s_t)\right]^{1-\sigma}}W_t(s_t)\label{5}
            \end{align}
            where \(s_t\in S_t\)
            \item Specific case 1\\
            From \(\eqref{4}\) and \(\eqref{5}\)
            \[\begin{dcases}
                p_1(s_1)=\frac{\beta}{2}\\
                c_{i,0}^\ast=c_{i,1}^\ast(s_1)=5
            \end{dcases}\quad i\in(A,B),s_1\in S_1\]
            Intuitively, consumer is indifferent between state \(s_1\) and state \(s'_1\) because there is no uncertainty in terms of aggregate endowment.
            \item The coeffcient of relative risk aversion is 
            \begin{align*}
                \gamma&=-\frac{cu''(c)}{u'(c)}\\
                &=-\frac{c(-\sigma)c^{-\sigma-1}}{c^{-\sigma}}\\
                &=\sigma
            \end{align*}
            The elasticity of intertemporal substitution is \(\dfrac{1}{\sigma}\), which can be viewed as the inverse of the coeffcient of relative risk aversion.
            The larger \(\sigma\) is, the less willing is the household to substitute consumption across time.
            Therefore, the elasticity of intertemporal substitution measures the smoothing incentive over time;
            the coefficient of relative risk aversion measures the smoothing incentive over state.
            \item Specific case 2\\
            \(B\) is better off because increasing probability \(\pi(s'_1)\) yields high benefit to one who owns larger endowment at state \(s'_1\), i.e. \(B\).\\
            From \(\eqref{4}\) and \(\eqref{5}\)
            \[\begin{dcases}
                p_1(s_1)=\frac{\beta}{3}\\
                p_1(s'_1)=\frac{2\beta}{3}\\
                c_{A,0}^\ast=c_{A,1}^\ast(s)=\frac{15+14\beta}{3(1+\beta)}\\
                c_{B,0}^\ast=c_{B,1}^\ast(s)=\frac{15+16\beta}{3(1+\beta)}
            \end{dcases}\quad s\in S_1\]
            From the results above, \(B\) gets a higher consumption fraction, which is consistent with the previous arguement. 
        \end{enumerate}
        \item Sequential trade
        \begin{enumerate}
            \item Given initial distribution of assets \(b_{i,-1}\), endowment processes and their corresponding probability distributions \(\{\{w_{i,t}(s_t)\}_{s_t\in S_t}\}_{t=0}^1\) for each \(i\),
            a competitive equilibrium with sequential trade is a set of allocations \(\{\{c_{i,t}^\ast(s_t),b_{i,t}^\ast(s_t)\}_{s_t\in S_t}\}_{t=0}^1\) 
            for each \(i\) and contingent claims prices \(\{\{q_{t}^\ast(s_t)\}_{s_t\in S_t}\}_{t=0}^1\) such that
        \begin{enumerate}
            \item Given the price system, the allocation solves each consumer's problem. For \(i=A,B\)
            \begin{align*}
                &\{c_{i,t}^\ast(s_t),b_{i,t}^\ast(s_t)\}_{t=0}^1=\argmax_{\{c_{i,t}(s_t),b_{i,t}(s_t)\}_{t=0}^1}\sum_{t=0}^1\sum_{s_t\in S_t}\beta^t\pi(s_t)\frac{[c_{i,t}(s_t)]^{1-\sigma}}{1-\sigma}\\
                &\begin{array}{r@{\quad}r@{}l@{\quad}l}
                s.t.&c_{i,t}(s_t)+\sum\limits_{s_t}q_{t}^\ast(s_t)b_{i,t}(s_t)&= b_{i,t-1}(s_t)+w_{i,t}(s_t)&\forall t\\
                &c_{i,t}(s_t)&\ge0 &\forall t\\
                &b_{i,-1}&=0\\
                &b_{i,1}(s_t)&=0
                \end{array} 
            \end{align*}
            \item All markets clear. For goods market 
            \begin{align*}
                &\sum_i c_{i,t}^\ast(s_t)=\sum_i w_{i,t}(s_t)&\forall t,\forall s_t\in S_t\\
                \intertext{For asset market}
                &\sum_i b_{i,t}^\ast(s_t)=0&\forall t,\forall s_t\in S_t
            \end{align*}
        \end{enumerate}
            \item Since date-0 equilibrium and sequential equilibrium are equivalent, from \(\eqref{4}\) and \(\eqref{5}\)
            \begin{align}
                q_{t-1}^\ast(s_t)&=p_t^\ast(s_t)=\beta^t\pi(s_t)\left[\frac{W_t(s_t)}{W_0}\right]^{-\sigma}\label{6}\\
                b_{i,t-1}^\ast(s_t)&=c_{i,t}^\ast(s_t)-w_{i,t}(s_t)\notag\\
                &=\frac{\sum\limits_{t}\sum\limits_{s_t}\beta^t\pi(s_t)\left[W_t(s_t)\right]^{-\sigma}w_{i,t}(s_t)}{\sum\limits_{t}\sum\limits_{s_t}\beta^t\pi(s_t)\left[W_t(s_t)\right]^{1-\sigma}}W_t(s_t)-w_{i,t}(s_t)\label{7}
            \end{align}
            where \(s_t\in S_t\)
            \item Specific cases
            \begin{enumerate}
                \item 
                \begin{enumerate}
                    \item 1c\\
                    From \(\eqref{6}\) and \(\eqref{7}\)
                    \[\begin{dcases}
                        q_0(s)=\frac{\beta}{2}\\
                        b_{A,0}^\ast(s_1)=b_{B,0}^\ast(s'_1)=-1\\
                        b_{A,0}^\ast(s'_1)=b_{B,0}^\ast(s_1)=1
                    \end{dcases}\quad s\in S_1\]
                    The cost of the portfolio bought by each consumer at time 0 is 0, they are neither net borrower or lender.
                    \item 1e\\
                    From \(\eqref{6}\) and \(\eqref{7}\)
                    \[\begin{dcases}
                        q_0(s_1)=\frac{\beta}{3}\\
                        q_0(s'_1)=\frac{2\beta}{3}\\
                        b_{A,0}^\ast(s_1)=-\frac{3+4\beta}{3(1+\beta)}\\
                        b_{A,0}^\ast(s'_1)=\frac{3+2\beta}{3(1+\beta)}\\
                        b_{B,0}^\ast(s_1)=\frac{3+4\beta}{3(1+\beta)}\\
                        b_{B,0}^\ast(s'_1)=-\frac{3+2\beta}{3(1+\beta)}
                    \end{dcases}\]
                    The cost of the portfolio bought by consumer A at time 0 is \(\dfrac{\beta}{3(1+\beta)}>0\), which turns out to be net lender.\\
                    The cost of the portfolio bought by consumer B at time 0 is \(-\dfrac{\beta}{3(1+\beta)}<0\), which turns out to be net borrower.
                \end{enumerate}
                \item For both specific cases in 1c and 1e, the equilibrium price of a discount bond is 
                \begin{align*}
                    p_b=\sum_{s_1}\pi(s_1)q_0(s_1)=\beta&&s_1\in S_1
                \end{align*}
            \end{enumerate}
            \item Given initial distribution of asset \(b_{i,-1}\), endowment processes and their corresponding probability distributions \(\{\{w_{i,t}(s_t)\}_{s_t\in S_t}\}_{t=0}^1\) for each \(i\),
            a competitive equilibrium with sequential trade is a set of allocations \(\{b_{i,t}^\ast,\{c_{i,t}^\ast(s_t)\}_{s_t\in S_t}\}_{t=0}^1\) 
            for each \(i\) and  risk free bond prices \(q_{b}^\ast)\) at \(t=0\) such that
                \begin{enumerate}
                    \item Given the price system, the allocation solves each consumer's problem. For \(i=A,B\)
                    \begin{align*}
                        &\{c_{i,t}^\ast(s_t),b_{i,t}^\ast\}_{t=0}^1=\argmax_{\{c_{i,t}(s_t),b_{i,t}\}_{t=0}^1}\sum_{t=0}^1\sum_{s_t\in S_t}\beta^t\pi(s_t)\frac{[c_{i,t}(s_t)]^{1-\sigma}}{1-\sigma}\\
                        &\begin{array}{r@{\quad}r@{}l@{\quad}l}
                        s.t.&c_{i,0}+q_{b}^\ast b_{i,0}&= w_{i,0}\\
                        &c_{i,1}(s_1)&=w_{i,1}(s_1)+b_{i,0}\\
                        &c_{i,t}(s_t)&\ge0 &\forall t\\
                        &b_{i,-1}&=0
                        \end{array} 
                    \end{align*}
                    \item All markets clear. For goods market 
                    \begin{align*}
                        &\sum_i c_{i,t}^\ast(s_t)=\sum_i w_{i,t}(s_t)&\forall t,\forall s_t\in S_t\\
                        \intertext{For asset market}
                        &\sum_i b_{i,t}^\ast=0&\forall t
                    \end{align*}
                \end{enumerate}
                FOC
                \[\begin{dcases}
                    (c_{i,0}^\ast)^{\sigma}=\lambda_{i,0}\\
                    \beta\pi(s_1)[c_{i,1}^\ast(s_1)]^{\sigma}=\lambda_{i,1}(s_1)\\
                    \lambda_{i,0}q_{b}^\ast=\sum_{s_1}\lambda_{i,1}(s_1)
                \end{dcases}\quad s_1\in S_1\]
                where \(\lambda_{i,0},\lambda_{i,1}(s_1)\) are multipliers on budget constraints.\\
                Intuitively, the bond price will be different because it cannot be traded over state to smooth each consumer's consumption.  
        \end{enumerate}
    \end{enumerate}
\end{document}