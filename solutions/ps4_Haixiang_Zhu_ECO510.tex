\documentclass{article}
\usepackage{mathtools}
\usepackage{amssymb}
\usepackage{amsthm}
\usepackage{graphicx}
\usepackage{enumitem}
\title{Assignment 4}
\date{Sepetember 19, 2020}
\author{Haixiang Zhu}
\begin{document}
    \maketitle
    \renewcommand{\arraystretch}{1.2}
    \begin{enumerate}
        \item 
        \begin{enumerate}
            \item Since $\alpha$ is the capital share of output and $\beta$ is the discout factor,$\alpha\in(0,1)$ and $\beta\in(0,1)$. Thus $\alpha\beta\in(0,1)$.
            \begin{align*}
                \lim_{T\to\infty}k_{t+1}&=\lim_{T\to\infty}\alpha\beta\frac{1-(\alpha\beta)^{T-t}}{1-(\alpha\beta)^{T-t+1}}k_t^\alpha\\
                &=\alpha\beta k_t^\alpha,&\forall t=0,1,\dots,T
            \end{align*}
        \item 
            \begin{enumerate}
                \item Arbitrary value is given to $\alpha$ and $\beta$ in Figure 1.
                \begin{figure}[h!]
                    \includegraphics[width=\linewidth]{4_1bi.png}
                    \caption{Optimal $k_{t+1}$ versus $k_t$ $(\alpha=0.47,\beta=0.866)$}
                \end{figure}
                Similar to Solow Model, there exists a steady state $k^\ast$. It is unique if the trivial solution $(k=f(k)=0)$ is not included. The dynamics of capital outside
                the steady state will follow the direction of arrows in the Figure 1. If $k_0$ is greater than $k^\ast$, the capital stock will keep decreasing until the steady state fulfills; if $k_0$ is less than $k^\ast$, the capital stock will
                keep increasing until the steady state meets. 
                \item At the steady state, 
                \begin{align*}
                    &k^\ast=\alpha\beta (k^\ast)^\alpha\\
                    \Rightarrow&k^\ast=(\alpha\beta)^{\frac{1}{1-\alpha}}\\
                    \Rightarrow&i^\ast=k^\ast=(\alpha\beta)^{\frac{1}{1-\alpha}}\\
                    \Rightarrow&c^\ast=f(k^\ast)-i^\ast=(\alpha\beta)^{\frac{1}{1-\alpha}}[(\alpha\beta)^{-1}-1]
                \end{align*}
                By the Solow model's golden rule,
                \begin{align*}
                    &{f}'(k^{gold})=\delta\\
                    \Rightarrow&k^{gold}=\alpha^{\frac{1}{1-\alpha}}\\
                    \Rightarrow&i^{gold}=k^{gold}=\alpha^{\frac{1}{1-\alpha}}\\
                    \Rightarrow&c^{gold}=f(k^{gold})-i^{gold}=\alpha^{\frac{1}{1-\alpha}}[\alpha^{-1}-1]
                \end{align*}
                Thus, at Solow model's golden rule steady state, the capital stock, investment, consumption and production aren't related to time preference rate $\beta$, while the C-K model are. 
                Plus, the golden saving rate is \(\alpha\) in Solow model, while in C-K model, the saving rate is \(\alpha\beta\).
                \item According to the question 2 of Problem Set 3,
                \begin{align*}
                    \lim_{T\to\infty}z_t&=\lim_{T\to\infty}\alpha\beta\frac{1-(\alpha\beta)^{T-t+1}}{1-(\alpha\beta)^{T-t+2}}\\
                    &=\alpha\beta,&\forall t=1,2,\dots,T+1
                \end{align*}
                Economic interpretation
                \begin{align*}
                     &\because\delta=1\\
                     &\therefore z_t=\frac{k_{t+1}}{k_t^\alpha}=\frac{i_t}{k_t^\alpha}=s
                \end{align*}
                \clearpage
                \begin{figure}[h!]
                    \includegraphics[width=\linewidth]{4_1biii.png}
                    \caption{Optimal $z_{t+1}$ versus $z_t$}
                \end{figure}
                Hence, $Z_t$ is the savings rate. As shown in Figure 2, the optimal saving rate (the steady state) equals
                to $\alpha\beta$. For any $z_0$ not equal to $\alpha\beta$, the savings rate $z_t$ will jump to $\alpha\beta$ and remain unchanged in the following periods. 
            \end{enumerate}
        \end{enumerate}
        \item Maximizing problem
        \begin{align*}
            & \max\quad\sum_{t=0}^\infty(\frac{1}{1+r})^t(D_t-E_t)\\
            &\begin{array}{r@{\quad}l}
                s.t.&D_t+I_t=F(K_t)+E_t-C(E_t)\\
                &K_{t+1}=(1-\delta)K_t+I_t\\
                &D_t\ge0\\
                &E_t\ge0\\
                &K_{t+1}\ge0\\
                &I_t\ge0\\
                &K_0\text{ is given}   
            \end{array}           
        \end{align*}
        Lagrangian Function
        \begin{multline*}
            L=\sum_{t=0}^\infty\{(\frac{1}{1+r})^t(D_t-E_t)+\lambda_t[F(K_t)+E_t-C(E_t)-D_t-I_t]\\
            +\eta_t[(1-\delta)K_t+I_t-K_{t+1}]+\nu_tD_t+\mu_tE_t+\theta_tK_{t+1}+\gamma_tI_t\}
        \end{multline*}
        \begin{enumerate}
            \item  Necessary Conditions\\
            FOC
            \begin{equation*}
                \left\{\begin{aligned}
                    &\frac{\partial L}{\partial D_t}=(\frac{1}{1+r})^t-\lambda_t+\nu_t=0,\text{ for }t=0,1,\dots,\infty\\
                    &\frac{\partial L}{\partial E_t}=-(\frac{1}{1+r})^t+\lambda_t(1-{C}'(E_t))+\mu_t=0,\text{ for }t=0,1,\dots,\infty\\
                    &\frac{\partial L}{\partial I_t}=-\lambda_t+\eta_t+\gamma_t=0,\text{ for }t=0,1,\dots,\infty\\
                    &\frac{\partial L}{\partial K_{t+1}}=\lambda_{t+1}{F}'(K_{t+1})+\eta_{t+1}(1-\delta)-\eta_t+\theta_t=0,\text{ for }t=0,1,\dots,\infty
                \end{aligned}\right.
            \end{equation*}
            Equality Constraints
            \begin{equation*}
                \left\{\begin{aligned}
                    &F(K_t)+E_t-C(E_t)-D_t-I_t=0,\text{ for }t=0,1,\dots,\infty\\
                    &(1-\delta)K_t+I_t-K_{t+1}=0,\text{ for }t=0,1,\dots,\infty
                \end{aligned}\right.
            \end{equation*}
            Inequality Constraints
            \begin{equation*}
                \left\{\begin{aligned}
                    &D_t\ge0,\text{ for }t=0,1,\dots,\infty\\
                    &E_t\ge0,\text{ for }t=0,1,\dots,\infty\\
                    &K_{t+1}\ge0,\text{ for }t=0,1,\dots,\infty\\
                    &I_t\ge0,\text{ for }t=0,1,\dots,\infty
                \end{aligned}\right.
            \end{equation*}
            Multipliers are non-negative
            \begin{equation*}
                \left\{\begin{aligned}
                    &\nu_t\ge0,\text{ for }t=0,1,\dots,\infty\\
                    &\mu_t\ge0,\text{ for }t=0,1,\dots,\infty\\
                    &\theta_t\ge0,\text{ for }t=0,1,\dots,\infty\\
                    &\gamma_t\ge0,\text{ for }t=0,1,\dots,\infty
                \end{aligned}\right.
            \end{equation*}
            Complementary Slackness
            \begin{equation*}
                \left\{\begin{aligned}
                    &\nu_tD_t=0,\text{ for }t=0,1,\dots,\infty\\
                    &\mu_tE_t=0,\text{ for }t=0,1,\dots,\infty\\
                    &\theta_tK_{t+1}=0,\text{ for }t=0,1,\dots,\infty\\
                    &\gamma_tI_t=0,\text{ for }t=0,1,\dots,\infty
                \end{aligned}\right.
            \end{equation*}
            TVC
            \[\lim_{T\to\infty}\lambda_Tk_{T+1}=0\]
            \item $C(E_t)=0,\forall E_t$
            \begin{enumerate}
                \item Based on the FOC of $[D_t]$ and $[E_t]$, we have
                \begin{equation*}
                    \left\{\begin{aligned}
                        &\nu_t=\lambda_t-(\frac{1}{1+r})^t\\
                        &\mu_t=(\frac{1}{1+r})^t-\lambda_t
                    \end{aligned}\right.
                    \Rightarrow \nu_t+\mu_t=0,\forall t
                \end{equation*}
                Since multipliers are non-negative,
                \begin{equation*}
                    \left\{\begin{aligned}
                        &\nu_t\ge0\\
                        &\mu_t\ge0
                    \end{aligned}\right.
                    \Rightarrow \nu_t=\mu_t=0,\forall t
                \end{equation*}
                By the complementary slackness, two non-negativity constraints will never bind.
                \item $K_{t+1}$ will never bind because $F(K_t)$ satifies the Inada Conditions and if $K_t=0$, $D_t-E_t=-I_t\le0$, which implies the firm's objective cannot be optimal. Thus, $K_{t+1}>0,\forall t$ and by complementary slackness, $\theta_t=0,\forall t$.\\
                As for investment non-negativity constraints, if $I_t=0$, $\lim_{t\to\infty}K_{t+1}=0$, a contradiction. Hence, $I_t>0,\forall t$ and by complementary slackness, $\gamma_t=0,\forall t$.\\
                Plugging $\lambda_t,\eta_t,\mu_t,\nu_t,\theta_t,\gamma_t,I_t$, the necessary conditions can be reduced as follows.
                \begin{equation*}
                    \left\{\begin{aligned}
                        &{F}'(K_{t+1})=\delta+r,\text{ for }t=0,1,\dots,\infty\\
                        &F(K_t)+E_t-D_t+(1-\delta)K_t-K_{t+1}=0,\text{ for }t=0,1,\dots,\infty
                    \end{aligned}\right.
                \end{equation*}
                Since ${F}'>0$ and ${F}''<0$, ${F}'$ must be a injective function, which implies that there exists a unique steady stade $K^\ast=(F')^{-1}(\delta+r)$, given $\delta$ and r.
                If initial $K_0<K^\ast$ and ${F}'(K_0)>\delta+r$, $K_{t+1}$ will keep increasing until $K_{t+1}=K^\ast$ and ${F}'(K_{t+1})$ will keep decreasing until ${F}'(K_{t+1})=\delta+r$; if initial $K_0>K^\ast$ and ${F}'(K_0)<\delta+r$, $K_{t+1}$ will keep decreasing until $K_{t+1}=K^\ast$ and ${F}'(K_{t+1})$ will keep increasing until ${F}'(K_{t+1})=\delta+r$.
                \item In standard Cass-Koopmans Model, \[F'(K^\ast)=\frac{1}{\beta}+\delta-1\]
                In this model, \[F'(K^\ast)=r+\delta\]
                Since the time preference \(\beta\) can be also wriiten as \(\frac1{1+r}\), the two model are equivalent in terms of steady state.\\
                Plus, in C-K model, utility function is concave, individual will smooth their consumption. Given \(K_0\), \(K_t\) will converge to the steady state gradually.
                Analogously, given \(K_0\) in this model, \(K_1\) will jump to \(K^\ast\) by adjusting \((D-E)\).
                \item Given a $K^\ast$, we have
                \begin{align*}
                    &\left\{\begin{aligned}
                        &{F}'(K^\ast)=\delta+r\\
                        &F(K^\ast)+E_t-D_t-\delta K^\ast=0
                    \end{aligned}\right.\\
                    &\Rightarrow D_t-E_t=F(K^\ast)-\delta K^\ast,\forall t
                \end{align*}
                From the equation system above, we cannot determine $D_t$ and $E_t$ separately.
            \end{enumerate}
            \item \(C(0)=0, 0\le C(E)<E,0<{C}'(E)<1,{C}''(E)>0\)
            \begin{proof}
                From the FOC of [$D_t$] and [$E_t$], we have
                \begin{align*}
                    &\left\{\begin{aligned}
                        &\nu_t=\lambda_t-(\frac{1}{1+r})^t\\
                        &\mu_t=(\frac{1}{1+r})^t-\lambda_t(1-{C}'(E_t))
                    \end{aligned}\right.\\
                    &\Rightarrow\nu_t+\mu_t=\lambda_t{C}'(E_t),\forall t
                \end{align*}
                Since $\lambda_t>0$ (multiplier of a equality constraint) and $0<{C}'(E)<1$, the RHS is positive. Thus $\nu_t+\mu_t>0,\forall t$.
                Because multipliers are non-negative, three possible situations are as follows.
                \begin{enumerate}[label=(\roman*)]
                    \item $\nu_t>0,\mu_t>0$\\ 
                    By complementary Slackness, $D_t=E_t=0,\forall t$.
                    \item $\nu_t>0,\mu_t=0$\\ 
                    By complementary Slackness, $D_t=0,E_t>0,\forall t$.
                    \item $\nu_t=0,\mu_t>0$\\ 
                    By complementary Slackness, $D_t>0,E_t=0,\forall t$.
                \end{enumerate}
            \end{proof}
        \end{enumerate}
    \end{enumerate}
    
\end{document}