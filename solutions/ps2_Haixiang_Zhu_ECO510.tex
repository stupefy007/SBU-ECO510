\documentclass{article}
\usepackage{mathtools}
\usepackage{amsthm}
\title{Assignment 2}
\date{Sepetember 7, 2020}
\author{Haixiang Zhu}
\begin{document}
\maketitle
\begin{enumerate}
    \item C-K Model: 3-period
    \begin{enumerate}
        \item Lagrangian Function:
        \begin{multline*}
            L=\sum_{t=0}^2\beta^tu(c_t,1-n_t)+\sum_{t=0}^2\lambda_t(k_t^\alpha n_t^{1-\alpha}-c_t-k_{t+1}+(1-\delta)k_t)\\
            +\sum_{t=0}^2\mu_tk_{t+1}+\sum_{t=0}^2\theta_t n_t+\sum_{t=0}^2\eta_t(1-n_t)+\sum_{t=0}^2\nu_tc_t
        \end{multline*}
        Necessary Conditions:
        \begin{enumerate}
            \item FOC:
            \begin{equation*}
                \left\{\begin{aligned}
                    &\frac{\partial L}{\partial c_t}=\beta^t\frac{\partial u(c_t,1-n_t)}{\partial c_t}-\lambda_t+\nu_t=0,\text{ for }t=0,1,2\\
                    &\frac{\partial L}{\partial k_{t+1}}=-\lambda_t+\lambda_{t+1}(\alpha k_{t+1}^{\alpha-1}n_{t+1}^{1-\alpha}+1-\delta)+\mu_t=0,\text{ for }t=0,1\\
                    &\frac{\partial L}{\partial k_3}=-\lambda_2+\mu_2=0\\
                    &\frac{\partial L}{\partial n_t}=\beta^t\frac{\partial u(c_t,1-n_t)}{\partial n_t}+\lambda_t(1-\alpha)k_t^\alpha n_t^{-\alpha}+\theta_t-\eta_t=0,\text{ for }t=0,1,2
                \end{aligned}\right.
            \end{equation*}
            \item Equality Constraints:
            \begin{equation*}
                k_t^\alpha n_t^{1-\alpha}-c_t-k_{t+1}+(1-\delta)k_t=0,\text{ for }t=0,1,2
            \end{equation*}
            \item Inequality Constraints:
            \begin{equation*}
                \left\{\begin{aligned}
                    &k_{t+1}\ge0,\text{ for }t=0,1,2\\
                    &n_t\ge0,\text{ for }t=0,1,2\\
                    &1-n_t\ge0,\text{ for }t=0,1,2\\
                    &c_t\ge0,\text{ for }t=0,1,2
                \end{aligned}\right.
            \end{equation*}
            \item Multipliers are non-negative:
            \begin{equation*}
                \left\{\begin{aligned}
                    &\lambda_t\ge0,\text{ for }t=0,1,2\\
                    &\mu_t\ge0,\text{ for }t=0,1,2\\
                    &\theta_t\ge0,\text{ for }t=0,1,2\\
                    &\eta_t\ge0,\text{ for }t=0,1,2\\
                    &\nu_t\ge0,\text{ for }t=0,1,2
                \end{aligned}\right.
            \end{equation*}
            \item Complementary Slackness:
            \begin{equation*}
                \left\{\begin{aligned}
                    &\mu_tk_{t+1}=0,\text{ for }t=0,1,2\\
                    &\theta_tn_t=0,\text{ for }t=0,1,2\\
                    &\eta_t(1-n_t)=0,\text{ for }t=0,1,2\\
                    &\nu_tc_t=0,\text{ for }t=0,1,2
                \end{aligned}\right.
            \end{equation*}
        \end{enumerate}
        \item Properties of production function $f(k_t,n_t)=k_t^\alpha n_t^{1-\alpha}$, for $t=0,1,2$:
        \begin{enumerate}
            \item Positive and diminishing marginal returns of capital and labor:
            \begin{proof}
                \begin{align*}
                    &f_1=\alpha k_t^{\alpha-1}n_t^{1-\alpha}>0,f_{11}=-\alpha(1-\alpha)k_t^{\alpha-2}n_t^{1-\alpha}<0\\
                    &f_2=(1-\alpha)k_t^{\alpha}n_t^{-\alpha}>0,f_{22}=-\alpha(1-\alpha)k_t^{\alpha}n_t^{-\alpha-1}<0
                \end{align*}
            \end{proof}
            \item Inada Condition:
            \begin{proof}
                \begin{align*}
                    &\lim_{k_t\to0}f_1(k_t,n_t)=\lim_{k_t\to0}\alpha k_t^{\alpha-1}n_t^{1-\alpha}=\infty\\
                    &\lim_{n_t\to0}f_2(k_t,n_t)=\lim_{n_t\to0}(1-\alpha)k_t^{\alpha}n_t^{-\alpha}=\infty
                \end{align*}
            \end{proof}
        \end{enumerate}
        Properties of utility function $u(c_t,1-n_t)=\dfrac{c_t^{1-\sigma}}{1-\sigma}+A\dfrac{(1-n_t)^{1-\gamma}}{1-\gamma}$, for $t=0,1,2$:
        \begin{enumerate}
            \item Positive and diminishing marginal utility of consumption and leisure:
            \begin{proof}
                \begin{align*}
                    &u_1=c_t^{-\sigma}>0,u_{11}=-\sigma c_t^{-\sigma-1}<0\\
                    &u_2=A(1-n_t)^{-\gamma}>0,u_{22}=-\gamma(1-n_t)^{-\gamma-1}<0
                \end{align*}
            \end{proof}
            \item Inada Condition:
            \begin{proof}
                \begin{align*}
                    &\lim_{c_t\to0}u_1(c_t,1-n_t)=\lim_{c_t\to0}c_t^{-\sigma}=\infty\\
                    &\lim_{(1-n_t)\to0}u_2(c_t,1-n_t)=\lim_{(1-n_t)\to0}=A(1-n_t)^{-\gamma}=\infty
                \end{align*}
            \end{proof}
        \end{enumerate}
        Consider the consumption non-negative constraints $c_t\ge0$, for $t=0,1,2$. These constraints cannot be binding because of the Inada condition $\lim\limits_{c_t\to0}u_1(c_t,1-n_t)=\infty$ and FOC of \([c_t]\).
        A very small increase of consumption would lead to a huge increase in utility. Therefore $c_t>0$, for $t=0,1,2$.
        By the complementary slackness conditions, it must be that $\nu_t=0$, for $t=0,1,2$.\\
        Similarly, the labor constraints $n_t\ge0$ and $1-n_t\ge0$ for $t=0,1,2$ will not bind in any period thanks to the Inada condition $\lim\limits_{n_t\to0}f_2(k_t,n_t)=\infty,\lim\limits_{(1-n_t)\to0}u_2(c_t,1-n_t)=\infty$ and FOC of \([n_t]\).
        By the complementary slackness conditions, it must be that $\theta_t=\eta_t=0$, for $t=0,1,2$.\\
        Finally, consider the captial non-negative constraints. $k_1\ge0$ and $k_2\ge0$ are not binding due to the Inada condition $\lim\limits_{k_t\to0}f_1(k_t,n_t)=\infty$ and FOC of \([k_{t+1}]\). In the last period, however, $k_3\ge0$ is binding. 
        By FOC, we have $\mu_2=\lambda_2=\beta^2u_1(c_2,1-n_2)>0$. By the complementary slackness condition, $k_3=0$.
        \item Properties of utility function $u(c_t,1-n_t)=\dfrac{c_t^{1-\sigma}}{1-\sigma}$, for $t=0,1,2$:
        \begin{enumerate}
            \item Positive and diminishing marginal utility of consumption:
            \begin{proof}
                \begin{align*}
                    &u_1=c_t^{-\sigma}>0\\
                    &u_{11}=-\sigma c_t^{-\sigma-1}<0
                \end{align*}
            \end{proof}
            \item The utility is irrelevant to the working hour.
            \begin{proof}
                \begin{equation*}
                    \frac{\partial u(c_t,1-n_t)}{\partial n_t}=0
                \end{equation*}
            \end{proof}
        \end{enumerate}
        The optimal choice of labor is $n_t=1$, for $t=0,1,2$. The only difference between (b) and (c) is whether \(1-n_t\ge0\) is binding or not.\\
        Intuitive explanation: In part (b), there is a trade-off between consumption and working. 
        However, the new utility function in part (c) is irrelevant to the working hour, which implies working increases production without reducing utility.
        Thus, the optimum is utilizing labor as much as possible.
    \end{enumerate}
	\item C-K Model: finite horizon
	\begin{enumerate}
        \item Lagrangian Function:
        \begin{multline*}
            L=\sum_{t=0}^{T}\beta^t\{u(c_t)+\lambda_t[f(k_t)-c_t-i_t]+\gamma_t[(1-\delta)k_t+i_t-k_{t+1}]+\mu_tk_{t+1}+\nu_tc_t\}
        \end{multline*}
        Necessary Conditions:
        \begin{enumerate}
            \item FOC:
            \begin{equation*}
                \left\{\begin{aligned}
                    &\frac{\partial L}{\partial c_t}=\beta^t[{u}'(c_t)-\lambda_t+\nu_t]=0,\text{ for }t=0,1,\dots,T\\
                    &\frac{\partial L}{\partial i_t}=\beta^t(-\lambda_t+\gamma_t)=0,\text{ for }t=0,1,\dots,T\\
                    &\frac{\partial L}{\partial k_{t+1}}=\beta^t[-\gamma_t+\mu_t+\beta\lambda_{t+1}{f}'(k_{t+1})+\beta\gamma_{t+1}(1-\delta)]=0,\text{ for }t=0,1,\dots,T-1\\
                    &\frac{\partial L}{\partial k_T}=\beta^T(-\lambda_T+\mu_T)=0
                \end{aligned}\right.
            \end{equation*}
            \item Inequality Constraints:
            \begin{equation*}
                \left\{\begin{aligned}
                    &f(k_t)-c_t-i_t\ge0,\text{ for }t=0,1,\dots,T\\
                    &(1-\delta)k_t+i_t-k_{t+1}\ge0,\text{ for }t=0,1,\dots,T\\
                    &k_{t+1}\ge0,\text{ for }t=0,1,\dots,T\\
                    &c_t\ge0,\text{ for }t=0,1,\dots,T
                \end{aligned}\right.
            \end{equation*}
            \item Multipliers are non-negative:
            \begin{equation*}
                \left\{\begin{aligned}
                    &\beta^t\lambda_t\ge0,\text{ for }t=0,1,\dots,T\\
                    &\beta^t\gamma_t\ge0,\text{ for }t=0,1,\dots,T\\
                    &\beta^t\mu_t\ge0,\text{ for }t=0,1,\dots,T\\
                    &\beta^t\nu_t\ge0,\text{ for }t=0,1,\dots,T
                \end{aligned}\right.
            \end{equation*}
            \item Complementary Slackness:
            \begin{equation*}
                \left\{\begin{aligned}
                    &\beta^t\lambda_t(f(k_t)-c_t-i_t)=0,\text{ for }t=0,1,\dots,T\\
                    &\beta^t\gamma_t[(1-\delta)k_t+i_t-k_{t+1}]=0,\text{ for }t=0,1,\dots,T\\
                    &\beta^t\mu_tk_{t+1}=0,\text{ for }t=0,1,\dots,T\\
                    &\beta^t\nu_tc_t=0,\text{ for }t=0,1,\dots,T
                \end{aligned}\right.
            \end{equation*}
        \end{enumerate}
        \item Consider consumption non-negative constraints $c_t\ge0,\forall t=0,1,\dots,T$.
        By Inada condition $\lim\limits_{c_t\to0}{u}'(c_t)=\infty$, $c_t=0,\forall t=0,1,\dots,T$ can never be optimal and it has to be the case that $c_t>0,\forall t=0,1,\dots,T$.
        By Complementary slackness, this implies that $\nu_t=0,\forall t=0,1,\dots,T$.\\
        For the resource constraints, by FOC and ${u}'(c_t)>0$, we have $\lambda_t={u}'(c_t)>0,\forall t=0,1,\dots,T$. Thus, by complementary slackness, the resource constraints are binding, i.e. $f(k_t)-c_t-i_t=0,\forall t=0,1,\dots,T$.\\
        For the capital accumulation constraints, we have $\lambda_t=\gamma_t>0,\forall t=0,1,\dots,T$ from FOC. Therefore, by complementary slackness, the capital accumulation constraints are binding, i.e. $(1-\delta)k_t+i_t-k_{t+1}=0,\forall t=0,1,\dots,T$.\\
        Eventually, consider the captial non-negative constraints. $k_{t+1}\ge0,\forall t=0,1,\dots,T-1$ is not binding because $f(\cdot)$ satifies Inada condition and FOC of \([k_{t+1}]\) goes to infinity at \(k_{t+1}=0\). Thus, by complementary slackness, $\mu_t=0,\forall t=0,1,\dots,T-1$.
        However, in the last period, we have $\mu_T=\lambda_T>0$ from FOC. Therefore, by complementary slackness, $k_{T+1}=0$.
        \item Rewriting the remaining equtions
        \begin{equation*}
            \left\{\begin{aligned}
                &\lambda_t={u}'(c_t),\text{ for }t=0,1,\dots,T\\
                &\lambda_t=\gamma_t,\text{ for }t=0,1,\dots,T\\
                &\gamma_t=\beta\lambda_{t+1}{f}'(k_{t+1})+\beta\gamma_{t+1}(1-\delta),\text{ for }t=0,1,\dots,T-1\\
                &f(k_t)-c_t-i_t=0,\text{ for }t=0,1,\dots,T\\
                &(1-\delta)k_t+i_t-k_{t+1}=0,\text{ for }t=0,1,\dots,T\\
                &k_{T+1}=0
            \end{aligned}\right.
        \end{equation*}
        Eliminating multipliers gives
        \begin{gather*}
            \left\{\begin{aligned}
                &c_t=f(k_t)+(1-\delta)k_t-k_{t+1},\text{ for }t=0,1,\dots,T\\
                &{u}'(c_t)=\beta[{f}'(k_{k+1})+(1-\delta)]{u}'(c_{t+1}),\text{ for }t=0,1,\dots,T-1\\
                &k_{T+1}=0
            \end{aligned}\right.\\
            \Rightarrow
            \left\{\begin{aligned}
            &\begin{split}
                {u}'[f(k_t)+(1-\delta)k_t-k_{t+1}]=&\beta[{f}'(k_{t+1})+(1-\delta)]\\
                &\cdot{u}'[f(k_{t+1})+(1-\delta)k_{t+1}-k_{t+2}],\forall t=0,1,\dots,T-1
            \end{split}\\
            &k_{T+1}=0
            \end{aligned}\right.
        \end{gather*}
    \end{enumerate}
\end{enumerate}
\end{document}